\section{Course structure}
{\bf Reminder Lectures, actually try to turn around ie why do we collide and motivate use of 4-vectors like that}
\begin{itemize}
\item[1] Relativistic kinematics JR
\begin{itemize}
\item 4-vectors $p_{\mu}=(E,p_i)$, Lorentz covariance/invariance, transformations, contractions JR
\end{itemize}
\item[1] Collisions JR
\begin{itemize}
    \item Examples in two-body pion decay ($\pi^0$ or $\pi^+$)
    \item Define centre-of-mass energy and use relativistic kinematics to explain pros and cons of fixed target vs beam-beam colliders JR
    \end{itemize}
\end{itemize}
\item[2] Feynman Diagrams plus revision (have seen already in \texttt{PHYS22040} in context of QED) take parts from section 9 of last years doc JR
\begin{itemize}
\item[3] Concept of antiparticle through two solutions of dispersion relation $E=\pm\sqrt{m^2c^4+p^2c^2}$. KP
\item[3-4] Could consider adding Dirac equation KP
\item[4] More on Dirac, Dirac picture: The negative solutionKP represents anti-particle states. Vacuum of filled infinite -ve energy states+Pauli exclusion stops spiralling of electron energy. 
\item (move elsewhere) Wilson cloud chamber and discovery of positron
\item (move elsenwhere) EM interactions excite electron create hole, motivate Feynman diagram within this context (positron is electron going backwards in time--I think this is too vague). Certainly tell them this picture is flawed. Require quantisation of EM for complete explanation
\end{itemize}
\end{itemize}

{\bf Range of Forces}
\begin{itemize}
\item [5] Give relativistic form of Schr\"odinger equation ie Klein-Gordon. Explain this is generalisation of Maxwells equation for massive field. Solution gives Yukawa potential $\phi=Q/4\pi\epsilon_{0}re^{-mcr/\hbar}$ KP
\item[5] EM is long range force (1/r). Massive mediator of interaction can give rise to short-range interaction KP
\item[5] Mention forces in nature and give relative coupling strenths (at a day-to-day energy scale). Explain why Weak force with massive gauge bosons is considered Weak (ie strong but short ranged) KP
\item[5] Show how propagators change in Feynman diagrams (2->2 scattering). Mention weak and colour charges but explain will be discussed later KP
\end{itemize}

{\bf Concepts of lifetime width and cross-section}
\begin{itemize}
\item[6] Derivation of rate of survival and average lifetime KP
\item[6] Use uncertainty principle $\Delta E\Delta t\geq \hbar/2$ to motivate concept of width $\Gamma=\hbar/\tau$ and partial width and branching fraction KP
\item[7] Define flux and cross-section, and phase space, Golden Rule KP
\end{itemize}

{\bf The strong force} KP
\begin{itemize}
\item[8] Hadrons and Evidence of Quarks  KP
\begin{itemize}
\item[8] Question: What holds nucleons in nucleus together?
\item[8] Yukawa: Field analogous to EM but short ranged. Scales involved require mass of new boson 1/6 of proton mass
\item[8] Powell+Brazilian Dude (Lattes): Discovery of pion could be transmitter of this force that holds nucleons in nucleus together
\item[9] People thought picture was complete, group protons/neutrons one doublet and pions one triplet 
where the pions where mediators (so introduce isospin here)
\item[9] Butler: Discovery of $K^0\to\pi\pi$ and $K\to\pi\pi\pi$ in cosmic rays gave rise to more ``strange" particles (unexpected). Proliferation of other hadrons through collider experiments
\end{itemize}

{\bf Eightfold way} KP
\begin{itemize}
\item[10]$\to$ Explanation of zoo through eight-fold way:
\item Proton and neutrons were known to have finite size (scattering experiments a-la Rutherford) first evidence of sub-structure ie quarks!
\item Postulate existence of 3 types (flavours) of quarks with fractional EM charge within hadrons. 
\item Tell them that hadron wave-function made up of spatial (angular momentum), flavour (quark type) and spin. 
\item Use Clebcsh-Gordan for spin decomposition of 3 spin 1/2 objects for ground state hadrons.
\item Postulate how to decompose direct product of 3 SU(2) and 3 SU(3) states into irreducible representations. No need for group theory just given them the number of states they predict
\item Requirement of Fermi-statistics means overall wavefunction anti-symmetric (aka Pauli exclusion). This severely limits number of 3 quark states you can produce. In fact a-lot fewer than what is observed!
\item {\bf Recent research penta-quarks!}
\item Key missing ingredient: A new type of charge! Colour!
\item Introduction of colour wave-function ie another SU(3) decomposition to a singlet state (by default anti-symmetric) means product spatial,spin and flavour wave-functions MUST be symmetric. Many more states become available. Explain spectrum and predict $\Omega$
\item This constitutes the first evidence of colour charge!!
\item Note mesons are easy as made up of quark and anti-quark so no spin statistics to worry about
\item $e^+e^-\to q\bar{q}$ and measurement of $R:=\frac{\sigma(ee\to\mathrm{hadrons})}{\sigma(ee\to\mu\mu)}$ provides strong direct experimental evidence
\item Remaining problem: Have never observed a $quark$ (ok top quark special explain why in terms of heavy top quark and time-scale of strong interaction)! Leads to asymptotic freedom. This can be explained by fact that gluon couples to itself with large strength (details of why next year). No mathematical proof as to whether colour charge automatically leads to asymptotic freedom.
\end{itemize}
\item Formation of jets (string model) briefly explain
\begin{itemize}
\item Can even take a simple QCD potential $V(r)=-\alpha/r+\beta r$. First term familiar EM like potential, second term represents asymptotic freedom. $\beta$ is large and inserting into Schr\"odinger equation makes it unsolvable perturbatively!
\item {\bf Try to relate to our own work $B\to DDK$ for example}
\end{itemize}
\item $e-p$ scattering and DIS. Is there time to discuss proton-PDFs?
\item[10] Try to keep to half a lecture to describe structure of proton 
\end{itemize}

{\bf The weak force} JR or, if week 4, KP
\begin{itemize}
\item[11] Strangeness not conserved. Lifetime of strange hadrons large JR
\item[11] Explained through weak force and Cabbibo angle (build into matrix later) JR
\end{itemize}


{\bf Evidence of charm} JR or, if week 4, KP
\begin{itemize}
\item[12] Suppression of $K^0\to\mu\mu$ vs $K^{+}\to\mu\nu$
\item[12] Give the Feynman diagram and tell them the rate and show that introduction of charm
gives almost 0 rate
\item[12] 3x3 matrix now...
\end{itemize}

{\bf C P and CP} JR
\begin{itemize}
\item[13] Symmetries conservation laws (you remember that...)
\item Give expressions of C and P eigenstates for mesons and baryons
\item[13] Madam Wu Parity and the Weak force
\item[13] CPV in the Kaon system
\item[14] Operation of CP complex conjugates weak coupling 
\item[14] Leads to $V_{CKM}$ through parameter counting and indirect evidence for $b$-quark
\end{itemize}

{\bf Neutrino oscillations} JR
\begin{itemize}
\item[14] Try to squeeze in somewhere
\end{itemize}

{\bf Mixing in $B$ hadron systems} JR
\begin{itemize}
\item[15] Build up as example for any mixing of meson states
\item[15] Give B0 and LHCbs Bs mixing
\item[15] What is needed measure these (this acts as intro to particle physics detectors)
\item[15] These give evidence for existence fo top quark (very heavy)
\end{itemize}

{\bf Higgs mechanism} KP if you want to give your magic lecture (happy to do it, too).
\begin{itemize}
\item[16] Hand-wavy description of mechanism
\item[16] Ways that was observed
\end{itemize}

{\bf Beyond the SM} JR
\begin{itemize}
\item[17] Cosmological observations and Hierarchy problem
\end{itemize}

{\bf Revision} KP+JR
\begin{itemize}
\item[18] Cosmological observations and Hierarchy problem
\end{itemize}


