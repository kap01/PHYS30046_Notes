\section{Natural units*}
\label{sec:naturalUnits}
We will usually use ``natural'' units with 
\[
c=\hbar=1.
\]
 Disclaimer: If inconsistencies are found in the notes in terms of missing/additional
 factors of $\hbar$ or $c$ please let me know. The person that spots the most errors
 will be in for a prize (nothing too special, more about the 
 feeling of satisfaction and pride...) 
 
 Amongst other things, this implies that time and length have the same
 units. From
\[
  E^2 = (cp)^2 + (mc^2)^2,
\]
 which simplifies to
\[
  E^2 = p^2 + m^2
\]
 in our system of units, we see that energy, momentum and mass
 have the same units. The relationship between wavelength and
 momentum of the photon:
\[
   p = \hbar \frac{2\pi}{\lambda}
\]
 implies that the units for momentum are inverse to those for
 distance. If we choose \units{GeV} as our energy unit, we get:
\begin{itemize}
  \item \units{GeV} as unit for mass, energy, momentum
  \item \units{GeV^{-1}} as unit for time and distance
\end{itemize}

The trick for translating from natural units, where $\hbar=c=1$, to
other systems, is to multiply any part of the expression/formula in
natural units with factors of $1$ expressed in terms of $\hbar$, $c$,
$c^2$, $\hbar c$, etc, until the expression has the correct units in
both systems. Remember that the units of $\hbar$ are [Energy]$\times$[Time] (e.g. \units{J\, s} or \units{MeV \, s} in SI units) while the units of $c$ are [Distance]/[Time]. Their product has units of [Energy]$\times$[Distance]. A relationship that is particularly useful for translating those units back to SI
 units is
\[
  \hbar c = \un{196}{MeV\,fm}.
\]
It's worth remembering this, or at least
\[
  \hbar c \approx \un{200}{MeV\,fm}.
\]

Let's for example translate the lifetime of a particle
from \units{MeV^{-1}} into seconds. The width of the \prt{\phi} particle is
$\Gamma = \un{4.458}{MeV}$, so, in our system of units, its lifetime
is $\tau = 1/\Gamma=\un{0.2243}{MeV^{-1}} =\un{0.2243}{MeV^{-1}} \hbar
= \un{0.2243}{MeV^{-1}} \hbar c/c$. The expression on the right is
exactly the same as the one on the left in our system of units, but it
also has the correct units in the SI system (time). Hence
 \begin{eqnarray*}
\tau &=&       \un{0.22}{MeV^{-1}} \hbar c/c                 \\
     &\approx& \un{0.22}{MeV^{-1}} \cdot \un{200}{MeV\,fm} /c      \\
     &\approx&       \un{44}{fm} /c                                \\
     &\approx& \un{44/3 \cdot 10^{-15}\cdot 10^{-8}}{s}  \\
     &\approx&        1.5 \cdot 10^{-22} \units{s}
\end{eqnarray*}

\exercise{
Translate the following recipe into natural units:
"Take 2 eggs, 500g of flour and a pint of milk. Mix for 10 min."
\vspace{1ex}\\
\rotatebox{180}{\parbox{0.9\textwidth}{
\begin{itemize}
\item 2 is just a number and remains invariant under the unit change.
\item 500$g$ is a mass, we want to know what that is in energy units, so we need to calculate: $mc^2 = \un{0.5}{kg}\cdot \un{9\cdot 10^{16}}{m^2/s^2} = \un{4.5\cdot 10^{16}}{J}$. Now multiply by $e/e$: $\un{4.5\cdot 10^{16}}{e J/e}$. Use $J/C = V$ and $e=\un{1.6 \cdot 10^{-19}}{C}$ to get
\un{2.8\cdot 10^{35}}{eV} of flour.
\item 1pint refers to a volume of \un{568}{cm^3}. We will measure volume in terms of $(eV)^-3$. We'll use that $\hbar c$ has units of [energy]$\times$[length]. So...
$\un{568\cdot 10^{-6}}{m^3} / (\hbar c)^3 = 
\un{5.68\cdot 10^{-4}}{m^3} / (\un{8\cdot 10^6}{MeV^3 fm^3})
=\un{0.71\cdot 10^{-4-6+45}}{MeV^{-3}} = \un{0.71\cdot 10^{35}}{MeV^{-3}}$
\item We want to multiply \un{600}{s} with factors of $\hbar$ and $c$ until we get inverse energy units. This is achieved by dividing by $\hbar$, which has units [energy]$\times$[time]. So we get $\un{600}{s} / \hbar$, which we calculate again with the $\hbar c$ trick, $\un{600}{s} /\hbar c/c = 600s /(\un{200}{MeV\,fm}) \cdot (3\cdot 10^8 m/s) = \un{9\cdot 10^{22}}{MeV^{-1}}$. 
\end{itemize}
So the solutions is:
"Take 2 eggs, a mass of \un{2.8\cdot{10^{35}}}{MeV} of flour and a volume of \un{0.71\cdot 10^{35}}{MeV^{-3}} of milk. Mix for \un{9\cdot 10^{22}}{MeV^{-1}}"
}}
}

Further discussion on natural units that goes beyond what is examinable
can be found in the Appendix.

