

\section{Introduction}
\subsection{Course content}
The goal of this course is to expand on ideas
introduced during your $2^{\rm nd}$ year Nuclear and Particle physics 
course regarding fundamental interactions in nature. Special focus will be given to how (non-)conservation of various quantum numbers gives rise to our understanding of the strong-, weak-nuclear and electromagnetic forces. Topics will include 
how our understanding of the strong force has evolved from the discovery of the pion and strong-isospin, to quarks, gluons and to modern particle physics. Will will also discuss how the measurements of quark transitions and Charge conjugation-Parity violation (CP violation), cemented our understanding of the weak force, and led to the prediction of new (and now familiar) particles. Finally, if time permits will touch on the role of the Higgs boson.

Key measurements which lead to breakthroughs in the field will be discussed, as well as key experimental techniques of particle detection.

\subsection{Course particulars}
The majority of the course will be taught through live lectures which will go through the majority of the material.
Some material will be taught through recordings. During the lectures you will have a chance to solve various exercises
and you are encouraged to ask questions on taught material both live and pre-recorded. 

In total the course consists of 13 live lectures and 3 two-hour live problems classes. 
Regarding reading material, the course notes should be self contained however you can find additional useful information in:
\begin{itemize}
\item Particle Physics (Martin and Shaw)
\item Introduction to elementary particles (Griffiths)
\item Modern Particle Physics (Thomson)
\end{itemize}
Drop-in sessions: \\
Fridays 11:00 to 12:00 during TB2 in my office 4.54.
If you would like a separate one-to-one meeting, please me send an email to arrange a time.

\paragraph{Exercises}: Throughout these notes you will find a set of exercises which are meant both to familiarise yourselves which the material, as well as to get a chance to learn something interesting, beyond the scope of the course. Exercises which are meant to go above and beyond the syllabus will be clearly marked.

%\paragraph{Online Quizes:} Two blackboard based quizes consisting of 10 multiple choice questions each worth 10\% of unit mark will be made available for you to complete during Week10 and Week12. 

\paragraph{Exam}: You will have 2.5 hours  to complete the entire exam. There will be 50 marks assosciated to the Particle Physics part of the Themes of Modern Physics C

\paragraph{This document} contains a few sections of further reading beyond the scope of the course. These are marked with an asterisk in the section title. While they are not required reading, they still provide useful information that should aid your understanding of the subject. You will also find extensive Appendices with additional non examinable material.
%%
