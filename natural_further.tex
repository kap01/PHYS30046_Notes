\section{What's natural about natural units?}
\label{sec:NaturalUnitsAppendix}

Especially those new to natural units often don't like them. Now, independent of
whether you happen to like them or not, you should probably get used to them
anyway, because if you want to communicate with other particle physicists,
you'll need to be able to use, or at least understand, natural units.  But I
think there's more to it.  Not only do natural units make your equations look
neater and therefore easier to follow, (keeping track of all those $\hbar$s
and $c$s can get quite tedious), but they are a consequence of real physical
insight. The reason we \emph{can} use the same units for space and time is
that space and time \emph{are}, to some degree, the same, and that is the
great insight of Special Relativity, and of Minkovsky space-time. It is not
just adding the $t$ to your usual 3-vector, a Lorentz boost really mixes those
time and space dimensions with each other.  And at the end it is
\[
c=\mathrm{constant},
\]
the experimental result that is at the very heart of Special Relativity, that
allows us to actually measure times in terms of the distance travelled by
light, or distances in units of time. The current definition of the good old
meter is based on how far light travels in a given amount of time. But if one
is defined in terms of the other anyway, why use two different units?

Apart from Relativity, the other big leap forward in the physics of the 20th
century was quantum mechanics, and one of its most important, and earlierst
equations is
\[
   p = \hbar \frac{2\pi}{\lambda}
\]
giving us a one-to-one relationship  between (inverse) distance and
momentum. Again it is the fundamental nature of the equation, and the
insight that $\hbar$ is a constant of nature, that allows us to measure
momenta in terms of inverse distances and vice versa - so why not give them
the same units? Similarly for energy and mass from
\[
 E = \gamma m c^2
\]
and energy and momentum from
\[
  E^2 = (cp)^2 + (mc^2)^2,
\]

Still not convinced? Take Newton's
\[
   ma = F = G \frac{Mm}{r^2}
\]
Note the repetition of the letter $m$. Each of the two instances
really means something quite different from the other: On the left
hand side it is the inertia of the body, i.e. its resistance to change
its velocity - the bigger $m$ is, the smaller is $a$ for the same
force $F$. We could call this $m_i$. On the right hand side, the $m$
stands for its power to attract another body, i.e. it is its
``gravitational charge" - it fulfils the same role as the electric
charge in Culoumb's law. We could call it $m_c$. It is
a great insight that for gravity, these two are proportional to each
other, $m_i \propto m_c$. And this is the reason we set the
proportionality constant between them to $1$, use the same units for
both, and even give them the same name - simply ``mass'', rather than
``inertial resistance'' and ``gravitational charge''. This seems
rather natural and the maths is far less clumsy - as with natural
units.
