\section{Neutrino Oscillations}
Neutrinos oscillate/mix, i.e. $\nu_e$ created in the sun via $p \to n e^+ \nu_e$ changes, in their long journey to earth, to $\nu_{\mu}$ or $\nu_{\tau}$; only about 1/3 arrive as $\nu_e$. Slightly faster are the oscillations that can be seen in $\nu_{\mu}$ that are created in the upper atmosphere (pion decay), which can change to $\nu_{\tau}$ (and, with a very small probability, $\nu_e$).

This implies that the flavour eigenstates (neutrinos created in reactions like $\pi^+ \to \mu^+ \nu_{\mu}$) are not the mass eigenstates of the neutrinos. These are (somewhat unimaginatively) referred to as $\nu_1, \nu_2, \nu_3$. The oscillation frequency is proportional to the mass differences between the neutrinos involved, which are very small.

The math of neutrino oscillations with three types of neutrinos is actually quite tricky. But it becomes quite easy, and more instructive, if we assume only two types of neutrinos. Then, the probability that a neutrino of type $\nu_{\alpha}$ (e.g. a $\nu_{\mu}$ created in the upper athmosphere) becomes a different type of neutrino $\nu_{\beta}$ (e.g. a $\nu_{\tau}$) is given by
\begin{equation}
    P(\nu_{\alpha} \to \nu_{\beta})
    =
    \sin^2(\theta) \sin^2\left( \frac{\Delta m^2 \,L}{4E} \right)
\end{equation}
where $L$ is the distance travelled, $E$ is the energy of the neutrino, and $\Delta m^2$ is the differences of the squared neutrino masses, $m_{\alpha}^2 - m_{\beta}^2$. The angle $theta$ is the mixing angle. For two neutrino generations, there would be only one (like the Cabibbo angle), for three generations there are three mixing angles. The equation above is in natural units, which is a bit cumbersome in this case, because we operate with numbers of very different scale. Another formulation of the same equation is:
\begin{equation}
    P(\nu_{\alpha} \to \nu_{\beta})
    =
    \sin^2(\theta) \sin^2\left( 1.25 \frac{\units{GeV\,km}}{\units{eV^2}}\frac{\Delta m^2 \,L}{4E} \right)
\end{equation}
So now you can plug in neutrino masses in \units{eV}, neutrino energies in \units{GeV} and distances in $km$.

To see neutrinos, we need huge detectors and huge numbers of neutrinos, because they do not like to interact. Billions of neutrinos pass through the earth every second unhindered.

The fact that neutrinos mix (in the same way as quarks) implies that there is a neutrino mixing matrix that could have a CP violating complex phase. This could eventually explain the baryon asymmetry of the universe through a process called leptogenesis.

The next generation of neutrino experiments will send high intensity neutrino beams hundreds of kilometers through the earth to enormous underground detectors, to look for CP violation in neutrinos. If you want to know more about these experiments, have a look at \href{http://www.dunescience.org/}{http://www.dunescience.org} and 
\href{http://www.hyperk.org/}{http://www.hyperk.org}. Or ask us.
