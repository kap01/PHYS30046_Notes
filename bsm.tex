\section{Beyond the Standard Model}
The Standard Model of particle physics (SM) has been incredibly successful and has withstood the most stringent experimental tests. So why look further?

There are a number or reasons why we are still not happy with the SM. They can be divided into two classes.
\begin{enumerate}
\item Real problems - discrepancies of the SM with reality.
\item Aesthetic problems - aspects of the SM we find unsatisfactory (but not necessarily wrong)
\end{enumerate}

A problems of the first type include
\begin{itemize}
\item The baryon asymmetry of the universe: There is more matter (and more baryons) in the universe than anti-baryons. The SM does not provide a mechanism for generating this asymmetry for a matter-antimatter symmetric start.
\item Dark matter: There is strong cosmological evidence fore dark matter, likely due to a fairly heavy, neutral particle. The SM does not provide a candidate for this. Many theories beyond the SM have a heavy, neutral, stable dark matter candidate.
\item Gravity: simply missing
\end{itemize}

Problems of the second type include:
\begin{itemize}
\item The Hierarchy problem - there are in fact many hierarchy problems, but \emph{the} hierarchy problem is the difference in scale between the energy scale where the e/w forces unify (around the W and Higgs mass, i.e. $10^2$ GeV), compared to the mass where gravity becomes important, which $~10^{19}$GeV. This turns out to be the same question why gravity is soo much weaker than then the other forces.
\item The fine tuning problem. This is closely related to the Hierarchy problem. "Naturally" one would expect the Higgs mass to be $~10^{19}$ in the SM. There are parameters in the SM that have to cancel to 11 digits in order to produce a Higgs mass with a "reasonable value" of $~10^2$ GeV. Such an exact cancellation, without any particular reason (such as a symmetry or so) is called fine tuning.
\item There are other, less dramatic hierarchy problems. Why are the masses between the particles so different (neutrinos $<1eV$, top $>10^7 eV$)? In general it would be nice to have a model that \emph{predicts} the particle masses.
\item The SM has too many free parameters (related to the previous point). The lepton masses, CKM parameters, PMNS parameters are all inputs to the theory, wouldn't it be nicer if they were outputs? Also, the structure seen in the mixing matrices suggest there should be some kind of underlying principle that explains them.
\end{itemize}

There are many "generic" ways to look for physics beyond the SM. For example looking for heavy, neutral particles that might be created at the LHC is motivated by the search for a dark matter candidate. These could be seen as missing mass in the detector. Some look for these particles in Higgs decays. The logic is simple: Since they are heavy, they are likely to couple strongly to the Higgs, even if they don't couple to much else. And finally, the "flavour physics" approach that led to the prediction of $c, b, t$ quarks. This focuses on decays whose properties (branching fraction, angular distribution of decay products, amount of CP violation etc) are precisely predicted in the SM. Any discrepancy hints at new physics beyond the SM. FCNC are particularly sensitive.


Supersymmetry is one of many theories beyond the SM that addresses some of these issues. It predicts that each fermion has a super symmetric partner that is a spin 0 boson (actually two: one for the left-handed and one for the right handed fermion). These partners are identical to the fermion, except they have have spin 0. And each boson (such as the photon) has a supersymmetric fermion as partner.

SUSY solves the hierarchy problem (we might sketch in the lectures, how). 

If SUSY is a good symmetry, it comes with a conserved quantity. This quantity is called R-parity. SUSY particles have R-parity $-1$ and SM particles have R-parity $+1$. If R-parity is conserved, the lightest SUSY particle cannot decay. If that is a neutral particle, it provides and excellent candidate for dark matter.

But SUSY particles have never been observed. This might be because SUSY particles are too heavy to be produced at the LHC. Now if SUSY were really a good symmetry, SUSY particles would have the same masses as SM particles. SUSY clearly is a broken symmetry. This is OK, in principle, but if it is too broken, it looses some if its appealing features (such as solving the hierarchy problems).

Another problem for SUSY is that many flavour physics measurements are highly sensitive to SUSY, even if SUSY particles are far heavier than particles that can be produced in the highest energy collisions at the LHC. For example the decay \prt{B_s \to \mu^+ \mu^-}, a flavour changing neutral current decay, could be substantially enhanced by SUSY particles whizzing around in the loop that mediates this decay. Alas, when we measured its branching fraction, we found exactly the value predicted by the SM.

So, SUSY might not be the true theory beyond the SM. All we know is that there is one. We're hoping that the true theory will be a big and beautiful surprise, and that we will see evidence for this surprise, soon.

Recent hints at LHCb in decays such as \prt{B^0 \to K^* \mu^+ \mu^-} (also a flavour changing neutral current), in the ratio of 
\prt{B^+ \to K^+ \mu^+ \mu^-} to \prt{B^+ \to K^+ e^+ e^-}, and in several other decay modes suggest that a flavour universality violating (i.e. coupling is different between $e, \mu, \tau$), flavour changing new type of $Z$ boson (called $Z'$) might play a role in these decays. It's early days, but it's quite exciting.

